\documentclass[aspectratio=169]{beamer}

\usepackage[utf8]{inputenc}
\usepackage{graphicx}
\usepackage{url}
\usepackage{listings}
\usepackage{xcolor}
\usepackage{minted}

\usetheme{metropolis}

\definecolor{codegreen}{rgb}{0,0.6,0}
\definecolor{codegray}{rgb}{0.5,0.5,0.5}
\definecolor{codepurple}{rgb}{0.58,0,0.82}
\definecolor{backcolour}{rgb}{0.95,0.95,0.92}

\lstdefinestyle{mystyle}{
    backgroundcolor=\color{backcolour},
    commentstyle=\color{codegreen},
    keywordstyle=\color{magenta},
    numberstyle=\tiny\color{codegray},
    stringstyle=\color{codepurple},
    basicstyle=\ttfamily\footnotesize,
    breakatwhitespace=false,
    breaklines=true,
    captionpos=b,
    keepspaces=true,
    numbers=left,
    numbersep=5pt,
    showspaces=false,
    showstringspaces=false,
    showtabs=false,
    tabsize=2
}

\lstset{style=mystyle}

\title{Speed-Dating Programming Languages}
\author{Joachim Schmidt}
\date{13. Oktober 2022}

\begin{document}

%\frame{\titlepage}

\begin{frame}{}
  \tableofcontents
\end{frame}

\section{Python}

\begin{frame}{}
  \begin{figure}
    \centering
    \includegraphics[width=\textwidth]{python-logo-generic.png}
    \caption{\cite{python_logo}}
    \label{fig:python_logo}
  \end{figure}
\end{frame}

\begin{frame}{Die Python Programmiersprache}
  \begin{itemize}
  \item interpretiert
  \item dynamisch typisiert
  \item große Auswahl an wissenschaftlichen Bibliotheken
  \end{itemize}
\end{frame}

\begin{frame}{Hello World}
  \inputminted[linenos]{python}{examples/hello.py}
\end{frame}

\begin{frame}{Dynamische Typisierung}
  \inputminted[linenos,fontsize=\scriptsize]{python}{examples/dynamic_typing.py}
\end{frame}

\begin{frame}{Nutzung in der scientific community}
  \inputminted[linenos]{python}{examples/dynamic_typing.py}
\end{frame}

\section{Java}

\begin{frame}{}
  \begin{figure}
    \centering
    \includegraphics[height=0.8\textheight]{Java_programming_language_logo.png}
    \caption{\cite{java_logo}}
    \label{fig:java_logo}
  \end{figure}
\end{frame}

\begin{frame}{Die Java Programmiersprache}
  \begin{itemize}
  \item objekt-orientiert
  \item läuft in einer virtuellen Maschine (JVM)
  \end{itemize}
\end{frame}

\section{Zig}

\begin{frame}{}
  \begin{figure}
    \centering
    \includegraphics[width=\textwidth]{zig-logo-dark.png}
    \caption{\cite{zig_logo}}
    \label{fig:zig_logo}
  \end{figure}
\end{frame}

\begin{frame}{Die Zig Programmiersprache}
  \begin{itemize}
  \item kompiliert zu Maschinencode
  \item low-level / manuelles Speichermanagement
  \item Ausführung von Code zur Kompilierzeit
  \end{itemize}
\end{frame}

\begin{frame}{Hello World}
  \inputminted[linenos]{zig}{examples/hello.zig}
\end{frame}

\begin{frame}{Ausführung von Code zur Kompilierzeit}
  \inputminted[linenos]{zig}{examples/comptime.zig}
\end{frame}

\begin{frame}{Quellen}
  \begin{thebibliography}{}
  \bibitem{python_logo}
    \url{https://www.python.org/static/community_logos/python-logo-generic.svg}

  \bibitem{zig_logo}
    \url{https://github.com/ziglang/logo/raw/master/zig-logo-dark.svg}

  \bibitem{java_logo}
    \url{https://upload.wikimedia.org/wikipedia/en/3/30/Java_programming_language_logo.svg}

  \end{thebibliography}
\end{frame}

\end{document}
