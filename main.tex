\documentclass[aspectratio=169]{beamer}

\usepackage[utf8]{inputenc}
\usepackage{graphicx}
\usepackage{url}
\usepackage{listings}
\usepackage{xcolor}
\usepackage{minted}
\usepackage{tikz}

\usetheme{metropolis}

\definecolor{codegreen}{rgb}{0,0.6,0}
\definecolor{codegray}{rgb}{0.5,0.5,0.5}
\definecolor{codepurple}{rgb}{0.58,0,0.82}
\definecolor{backcolour}{rgb}{0.95,0.95,0.92}

\title{Speed-Dating Programming Languages}
\author{Joachim Schmidt}
\date{13. Oktober 2022}

\begin{document}

% https://tex.stackexchange.com/questions/55806/mindmap-tikzpicture-in-beamer-reveal-step-by-step/55849#55849
%
% Keys to support piece-wise uncovering of elements in TikZ pictures:
% \node[visible on=<2->](foo){Foo}
% \node[visible on=<{2,4}>](bar){Bar}   % put braces around comma expressions
%
% Internally works by setting opacity=0 when invisible, which has the
% adavantage (compared to \node<2->(foo){Foo} that the node is always there, hence
% always consumes space plus that coordinate (foo) is always available.
%
% The actual command that implements the invisibility can be overriden
% by altering the style invisible. For instance \tikzsset{invisible/.style={opacity=0.2}}
% would dim the "invisible" parts. Alternatively, the color might be set to white, if the
% output driver does not support transparencies (e.g., PS)
%
\tikzset{
  invisible/.style={opacity=0},
  visible on/.style={alt={#1{}{invisible}}},
  alt/.code args={<#1>#2#3}{%
    \alt<#1>{\pgfkeysalso{#2}}{\pgfkeysalso{#3}} % \pgfkeysalso doesn't change the path
  },
}

%\frame{\titlepage}

%% \begin{frame}{}
%%   \tableofcontents
%% \end{frame}

\begin{frame}{}
  \begin{figure}
    \centering
    \includegraphics[width=\textwidth]{python-logo-generic.png}
    \caption{\cite{python_logo}}
    \label{fig:python_logo}
  \end{figure}
\end{frame}

\begin{frame}{Die Python Programmiersprache}
  \begin{itemize}
  \item interpretiert
  \item dynamisch typisiert
  \item große Auswahl an wissenschaftlichen Bibliotheken
  \end{itemize}
\end{frame}

\begin{frame}{Hello World}
  \inputminted[linenos]{python}{examples/hello.py}
\end{frame}

\begin{frame}{Bedingungen}
  \inputminted[linenos]{python}{examples/condition.py}
\end{frame}

\begin{frame}{Schleifen}
  \inputminted[linenos]{python}{examples/loops.py}
\end{frame}

\begin{frame}{Funktionen}
  \inputminted[linenos]{python}{examples/functions.py}
\end{frame}

\begin{frame}{Dynamische Typisierung}
  \inputminted[linenos,fontsize=\scriptsize]{python}{examples/dynamic_typing.py}
\end{frame}

\begin{frame}{Nutzung in der scientific community}
  \begin{tikzpicture}
    \node[left=0.1\textwidth, below=0.1\textwidth, visible on=<2->] at (current page.north west)
         { \includegraphics[width=0.3\textwidth]{NumPy_logo_2020.png} };
    \node[left=0.6\textwidth, visible on=<3->] at (current page.north west)
         { \includegraphics[width=0.2\textwidth]{scipy.png} };
    \node[left=0.4\textwidth, above=-0.1\textwidth, visible on=<4->] at (current page.north west)
         { \includegraphics[width=0.4\textwidth]{pytorch-logo-dark.png} };
  \end{tikzpicture}
\end{frame}

\begin{frame}{}
  \begin{figure}
    \centering
    \includegraphics[height=0.8\textheight]{Java_programming_language_logo.png}
    \caption{\cite{java_logo}}
    \label{fig:java_logo}
  \end{figure}
\end{frame}

\begin{frame}{Die Java Programmiersprache}
  \begin{itemize}
  \item objekt-orientiert
  \item läuft in einer virtuellen Maschine (JVM)
  \end{itemize}
\end{frame}

\begin{frame}{Hello World}
  \inputminted[linenos]{java}{examples/HelloWorld.java}
\end{frame}

\begin{frame}{Objekt-orientiert}
  \inputminted[linenos,fontsize=\scriptsize]{java}{examples/Inheritance.java}
\end{frame}

\begin{frame}{}
  \begin{figure}
    \centering
    \includegraphics[width=0.7\textwidth]{nim.png}
    \caption{\cite{nim_logo}}
    \label{fig:nim_logo}
  \end{figure}
\end{frame}

\begin{frame}{Die Nim Programmiersprache}
  \begin{itemize}
  \item kompiliert zu Maschinencode
  \item Python-ähnliche Syntax
  \item Makros für Meta-Programmierung
  \end{itemize}
\end{frame}

\begin{frame}{Hello World}
  \inputminted[linenos]{nim}{examples/hello.nim}
\end{frame}

\begin{frame}{Bedingungen}
  \inputminted[linenos]{nim}{examples/condition.nim}
\end{frame}

\begin{frame}{Schleifen}
  \inputminted[linenos]{nim}{examples/loops.nim}
\end{frame}

\begin{frame}{Syntax von Python und Pascal inspiriert}
  \inputminted[linenos]{nim}{examples/syntax.nim}
\end{frame}

\begin{frame}{Metaprogrammierung mit Makros und Templates}
  \inputminted[linenos]{nim}{examples/templates.nim}
\end{frame}

\begin{frame}{}
  \begin{figure}
    \centering
    \includegraphics[width=\textwidth]{zig-logo-dark.png}
    \caption{\cite{zig_logo}}
    \label{fig:zig_logo}
  \end{figure}
\end{frame}

\begin{frame}{Die Zig Programmiersprache}
  \begin{itemize}
  \item kompiliert zu Maschinencode
  \item low-level / manuelles Speichermanagement
  \item Ausführung von Code zur Kompilierzeit
  \end{itemize}
\end{frame}

\begin{frame}{Hello World}
  \inputminted[linenos]{zig}{examples/hello.zig}
\end{frame}

\begin{frame}{Bedingungen}
  \inputminted[linenos]{zig}{examples/condition.zig}
\end{frame}

\begin{frame}{Schleifen}
  \inputminted[linenos]{zig}{examples/loops.zig}
\end{frame}

\begin{frame}{Funktionen}
  \inputminted[linenos]{zig}{examples/functions.zig}
\end{frame}

\begin{frame}{Manuelles Speichermanagement}
  \inputminted[linenos]{zig}{examples/memory_management.zig}
\end{frame}

\begin{frame}{Ausführung von Code zur Kompilierzeit}
  \inputminted[linenos]{zig}{examples/comptime.zig}
\end{frame}

\begin{frame}{Quellen}
  \begin{thebibliography}{}
  \bibitem{python_logo}
    \url{https://www.python.org/static/community_logos/python-logo-generic.svg}

  \bibitem{zig_logo}
    \url{https://github.com/ziglang/logo/raw/master/zig-logo-dark.svg}

  \bibitem{java_logo}
    \url{https://upload.wikimedia.org/wikipedia/en/3/30/Java_programming_language_logo.svg}

  \bibitem{nim_logo}
    \url{https://raw.githubusercontent.com/nim-lang/assets/master/Art/logo-on-black.svg}

  \end{thebibliography}
\end{frame}

\end{document}
